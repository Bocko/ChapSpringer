%!TEX root = ChapSpringer_Main.tex
\subsection{Experimental set-up}
\subsubsection{Design and implemn}
we have carried out experiments using three different sub-blocks of the OpenSPARC-T2 SoC design: the core (SPC) that is gate dominated; the crossbar circuit (CCX) that is wire dominated; and the Ethernet module (RTX), an example of “typical” circuit. For graph formation hyper-edges have 10 different cost functions: Min-cut based (number of wires per C2C connection, average wirelength of all nets for each C2C connection, total WL of all nets in each C2C connection, product of 1 and 2, product of 1 and 3) and the inverse of the above. For clustering we have considered logical (four different hierarchical levels), top-down (two sizes) and newly developed bottom-up clustering methods (two sizes). For graph vertices weights we have been considering Area (as in the past) and area \& power.

Each design results in 80 points (8 clustering schemes X 10 cost functions per hyper-edge = 80 runs) that we analyze for key distribution parameters shown as box-plot (indicating Max/Med/Min and spread) for total and maximum wirelength. The following observations can be made based on the results obtained and shown on Figure a) below. a) Wire dominated circuits (crossbar) benefit more from 3D then the others (gains are higher then usual ~50% that we were observing in the past).  b) Gains (so the Total wirelength that will impact congestion and power and maximum wirelength that will impact performance) depend heavily on application and clustering/hyper-edge cost function. Finally, we observe high variability of the results; no cost function can be selected as clear winner. Thus, we need to run all the combination and then impellent methods and tools for robust solution space pruning.
 
Because the choice of cost functions (clustering) is not correlated with the gains across different designs, all combinations of clustering/hyper-edge costs are considered (here 80). From the design flow perspective this is not prohibiting, because the flow is fast (~1h for the above). To deal with such a huge design space we have implemented a method to choose Pareto dominant points only. To illustrate the benefits we have looked into the crossbar circuit and the initial search space of 80 solutions. If we consider simultaneously the following optimization objectives: minimum total wirelength, minimum critical path wirelength and maximum of 3D nets, by using the proposed solution we can reduce the space to 19 points that can be then handled manually. 

To better understand the impact of the 3D structure pitch we have compared: the total wirelength and the maximum wirelength as function of Number of 3D nets. This is shown on Figure b)(normalized to max Nº of 3D nets). Clearly a higher density of 3D nets will allow: less total wirelength, hence less congestion/power; finally less maximum wirelength means better performance. However the total and maximum benefits are converging. Finer grain partitioning should thus optimize other criteria.

As for power-aware partitioning, two problems had to be solved upfront. First, hyper-graph partitioning can handle only one weight per vertex/edge and linear combination of power/area will not work in this case. To solve this we have adopted normal graphs, since they can make use of more then 2 weights per graph vertex. Secondly, the graph partitioning (hyper and normal) can’t produce unbalanced partitions (e.g. max vertex weight in 1st and min in 2nd partition). To solve this problem we have introduced the following solution. A “Dummy node” is introduced in the graph. This node has (almost) zero area \& power that offsets the power budget to a certain value. This “Dummy node” is pre-assigned to the low-power die and will thus force unbalanced partition. To illustrate the impact of the power aware partitioning on other system parameters we have applied the proposed framework to OpenSPARC core shown on Figure c). With targeted power unbalance of 70-30% we can observe a slightly better total and maximum wirelength at expense of higher 3D net count & more important area unbalance (this is normal due to more constrained partitioning).

\subsubsection{3D system partitioning}
